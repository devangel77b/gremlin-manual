\documentclass{book}

\title{Flight Manual, Gremlin}
\author{Dennis Evangelista}
\date{\today}
  
\usepackage{natops_flight_manual}
  
\begin{document}
\maketitle
\frontmatter
\pagestyle{frontmatter}
% cover page
  % letter of promulgation - not doing
  % change junk - not doing
  % list of effective pages  - not doing
\tableofcontents
\listoffigures
% list of abbreviations

\chapter*{Preface}\index{preface}

\section*{Scope}
This NATOPS manual is issued by the authority of the Chief of Naval Operations and under the direction of Commander, Naval Air Systems Command in conjunction with the Naval Air Training and Operating Procedures Standardization (NATOPS) program. It provides the best available operating instructions for most circumstances, but no manual is a substitute for sound judgment. Operational necessity may require modification of the procedures contained herein. Read this manual from cover to cover. It’s your responsibility to have a complete knowledge of its contents.

\section*{Applicable publications}
The following applicable publications complement this manual:
\begin{itemize}
\item A1-GREMLIN-NFM-500 (Pilot's Pocket Checklist)
\end{itemize}

\section*{How to get copies}
Additional copies of this manual and changes thereto may be procured through the local supply system from NAVICP Philadelphia via DAAS in accordance with NAVSUP P-409 (MILSTRIP/MILSTRAP), or a requisition can be submitted to Naval Supply Systems Command via the Naval Logistics Library (NLL) website, www.nll.navsup.navy.mil. This publication is also available to view and download from the NATEC website, www.natec.navy.mil.

\section*{Automatic distribution (with Updates)}
This publication and changes to it are automatically sent to activities that are established on the Automatic Distribution Requirements List (ADRL) maintained by Naval Air Technical Data and Engineering Service Command (NATEC), in San Diego, CA. If there is continuing need for this publication, each activity’s Central Technical Publication Librarian must coordinate with the NATOPS Model Manager of this publication to ensure appropriate numbers of this and associated derivative manuals are included in the automatic mailing of updates to this publication.
\begin{note}
Activities not coordinating with the NATOPS Model Manager unit for more than 12 months may be dropped from distribution.
\end{note}

\section*{Updating the manual}
To ensure that the manual contains the latest procedures and information, NATOPS review conferences are held in accordance with the current OPNAVINST 3710.7 series.

\section*{Change recommendations}
Recommended changes to this manual or other NATOPS publications may be submitted by anyone in accordance with OPNAVINST 3710.7 series.
Change recommendations of any nature, (URGENT/PRIORITY/ROUTINE) should be submitted directly to the Model Manager via the NATOPS website (https://natops.navair.navy.mil) and the AIRS (Airworthiness Issue Resolution System) database. The AIRS is an application that allows the Model Manager and the NATOPS Office, Naval Air Systems Command (NAVAIR) AIR-4.0P to track all change recommendations with regards to NATOPS products. The Model Manager will be automatically notified via email when a new recommendation is submitted. A classified side of the website is available to submit classified information via the SIPRNET.

The address of the Model Manager of this aircraft/publication is:
\begin{verse}
Commanding Officer\\
ATTN: SH-60B NATOPS Model Manager HSL-40\\
P.O. Box 280118\\
Jacksonville, FL 32228-0118\\
Commercial 904-270-6322 x222 \\
DSN 960-6332 x222\\
\end{verse}

\section*{Your responsibility}
NATOPS manuals are kept current through an active manual change program. Any corrections, additions, or constructive suggestions for improvement of its contents should be submitted by urgent, priority or routine change recommendations, as appropriate.

\section*{NATOPS flight manual interim changes}
Interim changes are changes or corrections to NATOPS manuals promulgated by CNO or COMNAVAIRSYSCOM. Interim changes are issued either as printed pages, or as a naval message. The Interim Change Summary page is provided as a record of all interim changes. Upon receipt of a change or revision, the custodian of the manual should check the updated Interim Change Summary to ascertain that all outstanding interim changes have been either incorporated or canceled; those not incorporated shall be recorded as outstanding in the section provided.

\section*{Change symbols}
\fancypagestyle{rev1}[frontmatter]{%
\fancyfoot[RO,LE]{\sffamily\bfseries REV 1}
}
\thispagestyle{rev1}
\begin{changebar}
Revised text is indicated by a black vertical line in either margin of the page, like the one printed next to this paragraph. The change symbol shows where there has been a change. The change might be material added or information restated. A change symbol in the margin by the chapter number and title indicates a new or completely revised chapter.
\end{changebar}
  
\section*{Warnings, cautions, and notes}
The following definitions apply to WARNINGs, CAUTIONs, and Notes found throughout the manual.
\begin{warning}
An operating procedure, practice, or condition, etc., that may result in injury or death, if not carefully observed or followed.
\end{warning}
\begin{caution}
An operating procedure, practice, or condition, etc., that may result in damage to equipment, if not carefully observed or followed.
\end{caution}
\begin{note}
An operating procedure, practice, or condition, etc., that is essential to emphasize.
\end{note}

\section*{Wording}
The concept of word usage and intended meaning adhered to in preparing this manual is as follows:
\begin{enumerate}
\item ``Shall'' has been used only when application of a procedure is mandatory.
\item ``Should'' has been used only when application of a procedure is recommended.
\item ``May'' and “need not” have been used only when application of a procedure is optional.
\item ``Will'' has been used only to indicate futurity, never to indicate any degree of requirement for application of a procedure.
\item ``Land immediately'' is self-explanatory.
\item ``Land as soon as possible'' means land at the first site at which a safe landing can be made.
\item ``Land as soon as practical'' means extended flight is not recommended. The landing and duration of flight is at the discretion of the pilot in command.
\end{enumerate}
\begin{note}
This manual shall be carried in the aircraft at all times.
\end{note}




\mainmatter
\cleardoublepage
\part{The aircraft}

\cleardoublepage
\chapter{General description}
\begin{warning}
Beware of props, rotors, and jet wash.
\end{warning}

\begin{caution}
Connecting power to ground will damage the power supply. 
\end{caution}

\begin{note}
Happy and glad are synonyms. 
\end{note}

\section{The quadrotor}
\section{Dimensions}
\section{The motors}
\section{General arrangement}
\subsection{Exterior arrangement}
\subsection{Interior arrangement}


\chapter{Systems}

\part{Indoctrination}
\chapter{Aircrew training and qualifications}

\part{Normal procedures}
\chapter{Flight preparations}
\chapter{Normal procedures}
%\chapter{Special procedures}
\chapter{Functional checkflight procedures}

%\part{Flight characteristics}

% This part should be barberpole!
\cleardoublepage
\emergencymatter
\part{Emergency procedures}
\chapter{Emergency procedures}
%\chapter{Rotor injury}\index[emergency]{rotor injury}
%\chapter{Loss of FPV}\index[emergency]{loss of fpv}
%\chapter{Loss of control}\index[emergency]{loss of control}
%\chapter{Crash}\index[emergency]{crash}
%\chapter{Battery fire}\index[emergency]{battery fire}

\cleardoublepage
\mainmatter
%\part{Communications equipment and procedures}
%\chapter{Communications equipment and procedures}

%\part{Mission systems}
%\chapter{Mission systems}

\part{Crew resource management}
\chapter{Crew resource management}
\chapter{Evaluation}

\part{Performance data}
\chapter{Performance data}

\printindex

% back cover
% crosswind chart at end
%TO USE, DETERMINE WIND ANGLE RELATIVE TO RUN- WAY HEADING. ENTER CHART FROM LOWER LEFT ALONG WIND ANGLE LINE TO REPORTED VELOCITY ARC. FROM (INTERSECTION OF WIND ANGLE AND VELOCITY DETERMINE HEADWIND COMPONENT BY PROJECTING HORIZONTALLY TO LEFT; DETERMINE CROSSWIND COMPONENT BY DROPPING VERTICALLY TO BASELINE.

\end{document}