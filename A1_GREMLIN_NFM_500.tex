\documentclass{book}

\title{Pocket Checklist, FT Gremlin}
\author{Dennis Evangelista}
\date{\today}

% set page size 5.5 x 8.5
% use helvetica

% warning, caution, note; redefine warning and caution to get correct boxes (TODO)

% used for various indices
\usepackage[splitindex]{imakeidx}
\makeindex
\makeindex[name=misc,title=Miscellaneous data index]
\makeindex[name=emergency,title=Emergency procedures index]
% indices should be 1 column, helvetica all caps 

% for chapter toc
\usepackage{minitoc}
% chapter toc should be 1 column, helvetica, named "index".. 

\begin{document}
\maketitle
% cover page
\printindex[misc] % miscellaneous data index using imakeidx
% ttitle page
  % list of effective pages  - not doing
  % change summary - not doing
\printindex[emergency] % emergency procedures index using imakeidx
  % warning advisory cautions light - not doing
% for chapter toc
\dominitoc
\faketableofcontents

% emergency procedure tabs
% pages have barberpole on top
\chapter{Rotor injury}\index[emergency]{rotor injury}
\minitoc % chapter table of contents called index use minitoc

Hello world. 

\chapter{Loss of FPV}\index[emergency]{loss of fpv}
\minitoc

Hello world. 
\section{In flight}\index[emergency]{loss of fpv!in flight}
\section{On deck}\index[emergency]{loss of fpv!on deck}

\chapter{Loss of control}\index[emergency]{loss of control}
\minitoc
\section{In flight}\index[emergency]{loss of control!in flight}
\section{On deck}\index[emergency]{loss of control!on deck}

\chapter{Crash}\index[emergency]{crash}
\minitoc

\chapter{Battery fire}\index[emergency]{battery fire}
\minitoc

% no barberpole
\chapter{Reference data}
\minitoc % chapter table of contents called "reference data"
\section{Administrative brief}
\section{Preflight checklist}
\section{Takeoff checklist}
\section{Landing checklist}
\section{Postlanding}
\section{On-deck, maintenance troubleshooting}

% no barberpole, fold out pages
\chapter{System diagrams}
\index[misc]{system diagrams}
\index{system diagrams}

% back cover
% crosswind chart at end
%TO USE, DETERMINE WIND ANGLE RELATIVE TO RUN- WAY HEADING. ENTER CHART FROM LOWER LEFT ALONG WIND ANGLE LINE TO REPORTED VELOCITY ARC. FROM (INTERSECTION OF WIND ANGLE AND VELOCITY DETERMINE HEADWIND COMPONENT BY PROJECTING HORIZONTALLY TO LEFT; DETERMINE CROSSWIND COMPONENT BY DROPPING VERTICALLY TO BASELINE.

\end{document}